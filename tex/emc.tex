\documentclass{article}
\usepackage{graphicx} % Required for inserting images
\usepackage{caption}
\usepackage{enumerate}
\usepackage{hyperref}
\usepackage{titlesec}

\setcounter{secnumdepth}{4}
\setcounter{tocdepth}{4}
\title{Brandon Francis NEA}
\author{Brandon Francis}
\date{September 2025}
\usepackage{blindtext}
\usepackage{titlesec}
\begin{document}

\maketitle
\section*{Abstract}
\addcontentsline{toc}{chapter}{Abstract}
This report summarises my exploration into the phenomenon of synchronisation of oscillators. I have taken an abstraction of multiple . Our final model is based on a traffic grid in Chicago and uses multiple different timing systems to try to reduce traffic. Our results are then represented using a GUI that updates during run time and can be saved as an mp4 file. 


% Acknowledgements
\chapter*{Acknowledgements}
\addcontentsline{toc}{chapter}{Acknowledgements}

I would like to thank the organisers of the Exeter maths school, Dr Ed Horncastle \& Dr Aeran Fleming. I would also like to thank ...

% Table of Contents
\tableofcontents
\newpage

% Chapters
\chapter{Introduction}
\section{Problem Statement}

One of the most interesting physics phenomina is the synchronisation of interconnected networks. This phenomenon extends far beyond simple mechanical oscillators and manifests in numerous natural and engineered systems, from fireflies flashing in unison to power grid stability and even pedestrian footbridge dynamics. Understanding the mathematical principles behind synchronisation has always been an idea that I have wanted to experiment with for a long time.

\section{Scope}
In this study, I will examine different models to describe the connections between different oscillators and how they can be applied to both the power grid and human movement

\begin{enumerate}

    \item What are the effects of the number of oscillators and the coupling strength on the time taken to synchronise?

    \item How does introducing a reactionary delay affect the time to synchronise?
    \item Is there a way to predict if a non-zero coupling strength will eventually cause synchronisation?
    \item How does network topology affect synchronisation dynamics?
\end{enumerate}

Overall, the project is designed to provide insights into the nature of synchronisation within dynamic systems and identify conditions needed to produce or reduce synchronisation.

\section{Objectives}

The primary objective of this project is to build a robust simulation environment that accurately models synchronisation equipped with different network systems. The intended outcomes and objectives are as follows:

\begin{enumerate}
    \item Simulation Development: Create a discrete event simulation that mimics the behaviour of synchronising systems. The simulation will include differing connections of networks and include time delays.

    \item Comparison of Network types: Implement full, star and ring networks. Evaluate how each system influences final synchronisation, time to synchronise and.

    \item Performance Metrics Analysis: Collect and analyse data on critical performance metrics for each timing strategy. The main metrics include average wait time per vehicle, the number of junction passes, and the overall traffic throughput.

    \item Adaptive System Design: Develop and evaluate an adaptive traffic signal system that can adjust timings under various traffic conditions. The adaptive system shall be flexible and able to alter its parameters to achieve maximum network efficiency.

    \item Guidance for Urban Traffic Management: Ultimately, the project aims to provide insights that can inform urban traffic management decisions, highlighting how particular timing strategies can mitigate congestion and improve the performance of urban road networks. 
\end{enumerate}

\chapter{Background and Review of Literature}
\section{Related Work}

The study of synchronization in oscillator networks has a long and rich history, beginning with Huygens’ observation of pendulum clocks aligning their swings. Modern research has extended this phenomenon to complex networks, where synchronization plays a critical role in both natural and engineered systems. 

\subsection{Synchronization in Oscillator Networks}
Synchronization of coupled oscillators is a fundamental process in physics and engineering. Recent work has emphasized not only the stability of synchronous states but also the transient dynamics leading to synchrony. Nazerian et al. (2024) introduced the concept of \textit{transverse reactivity}, a metric that quantifies the instantaneous rate of growth or decay of desynchronizing perturbations, offering new insights into how networks converge to synchrony \cite{nature2024sync}. This highlights the importance of coupling strength and network topology in determining synchronization efficiency.

\subsection{Time Delays and Network Topology}
Time delays are inherent in real-world systems, from communication networks to traffic grids. Studies on mutually coupled oscillators with delays show that synchronization can be maintained if delays are properly accounted for in the system design \cite{ieee2022timedelays}. Moreover, distributed control approaches have been proposed to stabilize oscillator networks under non-uniform conditions, demonstrating that communication topology strongly influences synchronization outcomes \cite{mozafari2013distributed}.

\subsection{Applications in Power Grids}
Large-scale power systems are a prominent example where synchronization is vital. Research on smart grids has shown that synchronization depends on coupling parameters, load profiles, and network topology. Loss of synchrony can trigger cascading failures, underscoring the need for robust synchronization strategies in critical infrastructure \cite{dorfler2012smartgrids}. These findings provide a theoretical foundation for applying oscillator models to traffic systems, where stability and robustness are equally important.

\subsection{Summary}
The literature demonstrates that synchronization is a universal phenomenon with applications ranging from biological systems to engineered infrastructures. Key factors influencing synchronization include coupling strength, network topology, and time delays. In traffic networks, synchronization of signals can significantly reduce congestion and improve throughput. Building on these insights, this project aims to develop simulation models that integrate oscillator synchronization principles into urban traffic management.

\section{}

\bibliographystyle{plain}
\bibliography{references}

\end{document}
