\documentclass{report}
\usepackage{graphicx} % Required for inserting images
\usepackage{caption}
\usepackage{enumerate}
\usepackage{hyperref}
\usepackage{titlesec}
\setcounter{secnumdepth}{4}
\setcounter{tocdepth}{4}
\title{Brandon Francis NEA}
\author{Brandon Francis}
\date{September 2025}
\usepackage{blindtext}
\usepackage{titlesec}
\usepackage{amsmath}
\usepackage{amssymb}
\begin{document}
% Acknowledgement
\chapter*{Acknowledgements}
\addcontentsline{toc}{chapter}{Acknowledgements}

I would like to thank the organisers of the Exeter maths school, Dr Ed Horncastle \& Dr Aeran Fleming. I would also like to thank ...

% Table of Contents
\tableofcontents

% Chapters
\chapter{Introduction}
\section{Problem Statement}

One of the most interesting physics phenomina is the synchronisation of interconnected networks. This phenomenon extends far beyond simple mechanical oscillators and manifests in numerous natural and engineered systems, from fireflies flashing in unison to power grid stability and even pedestrian footbridge dynamics. Understanding the mathematical principles behind synchronisation has always been an idea that I have wanted to experiment with for a long time.

\section{Scope}
In this study, I will examine different models to describe the connections between different oscillators and how they can be applied to both the power grid and human movement

\begin{enumerate}

	\item What are the effects of the number of oscillators and the coupling strength on the time taken to synchronise?
	\item How does introducing a reactionary delay affect the time to synchronise?
	\item Is there a way to predict if a non-zero coupling strength will eventually cause synchronisation?
	\item How does network topology affect synchronisation dynamics?
\end{enumerate}

Overall, the project is designed to provide insights into the nature of synchronisation within dynamic systems and identify conditions needed to produce or reduce synchronisation.

\section{Objectives}

The primary objective of this project is to build a robust simulation environment that accurately models synchronisation equipped with different network systems. The intended outcomes and objectives are as follows:

\begin{enumerate}
	\item Simulation Development: Create a discrete event simulation that mimics the behaviour of synchronising systems. The simulation will include differing connections of networks and include time delays.
	      
	\item Comparison of Network types: Implement all-to-all, star and ring networks. Evaluate how each system influences final synchronisation, time to synchronise and.
	      
	\item Performance Metrics Analysis: Collect and analyse data on critical performance metrics for each timing strategy. The main metrics include average wait time per vehicle, the number of junction passes, and the overall traffic throughput.
	      
	\item Adaptive System Design: Develop and evaluate an adaptive traffic signal system that can adjust timings under various traffic conditions. The adaptive system shall be flexible and able to alter its parameters to achieve maximum network efficiency.
	      
	\item Guidance for Urban Traffic Management: Ultimately, the project aims to provide insights that can inform urban traffic management decisions, highlighting how particular timing strategies can mitigate congestion and improve the performance of urban road networks.
\end{enumerate}

\end{document}
